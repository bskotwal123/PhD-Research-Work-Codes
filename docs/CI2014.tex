
\documentclass[aps,prd,preprint,nofootinbib]{revtex4}
\usepackage{graphicx,amssymb,amsmath}
\usepackage[svgnames]{xcolor}


\begin{document}

\newcommand{\pT}{\textrm{$p_\textrm{T}$}}
\newcommand{\pTprime}{\textrm{$p_\textrm{T}^\prime$}}


\title{Search for Contact Interactions with Inclusive Jets\\Notes}

\author{Suman Beri$^1$, Suneel Dutt$^2$, and Harrison B. Prosper$^3$}

\affiliation{
$^1$Department of Physics, Panjab University, India\\
$^2$Department of Physics, Shoolini University, India\\
$^3$Department of Physics, Florida State University, USA
}


\date{\today}

\maketitle

\centerline{\large{CMS Collaboration}}

\section{Introduction}
The goal of this work is first to extend the 7 TeV search for contact interactions~\cite{bib:7TeVCI} to 8 TeV and
to prepare the groundwork for a search at 13 TeV.
For the 8 TeV search, we shall use the measured inclusive jet \pT\  spectrum at 8 TeV, which has the following
characteristics: 

\begin{itemize}
\item Jets defined by anti-$k_\text{T}$ algorithm with distance parameter $D = 0.7$
\item Jet $|y| < 0.5$
\item Jet \pT\ binning, range $507 \leq \pT  \leq 2500$ GeV
\end{itemize}
\noindent
For the QCD and CI models, we shall use the following:
\begin{itemize}
\item Renormalization ($\mu_r$) and factorization ($\mu_f$) scales with nominal value $\mu = $ jet $p_\text{T}$ and each scale changed independently by the factors $1/2$, 1, and 2, nine 
pairs in total.

\item PDFs: {\tt CT10nlo}, {\tt NNPDF23\_nlo\_as\_0118}, and {\tt MSTW2008nlo68cl}. 
\end{itemize}

Spectra will be computed for random samplings of PDFs from each PDF set, 100 PDF members per set.  Ensembles already exist for NNPDF, but for CTEQ and MSTW the ensembles shall be generated using the procedure described in Ref.~\cite{bib:MSTW} and released in version 6 of {\tt LHAPDF}. For each randomly sampled PDF member, we shall compute the (smeared) inclusive jet $p_\textrm{T}$ spectrum.
In order to maintain the correlations between the QCD and CI models, induced by the PDFs, the same set of sampled PDFs will be used for both. 

The data will be interpreted using the contact interaction (CI) model defined by the effective
Lagrangian~\cite{bib:Gao},
\begin{equation}
	L = 2 \pi \lambda \, \sum_{i=1}^6 \kappa_i \, O_i,
	\label{eq:L}
\end{equation}
where $\Lambda = \lambda^{-1/2}$ is the CI mass scale, $\kappa_i$~\footnote{We use $\kappa_i$ instead of $\lambda_i$, which is the notation used in Ref.~\cite{bib:Gao}, in order to avoid possible confusion with the parameter $\lambda$.} are constants and $O_i$ are dim-6 operators. Clearly,
the only unique combinations are the ratios $2\pi \kappa_i / \Lambda^2$; it is therefore only a matter of convention, and convenience, that in Eq.~(\ref{eq:L}) we split these ratios into an overall mass scale $\Lambda$ and the dimensionless parameters $\kappa_i$.
This model is defined by seven parameters: $\Lambda$, $\kappa_1,\cdots, \kappa_6$. In practice,  we shall follow Ref.~\cite{bib:CMSCI12} and consider specific combinations
of values for the $\kappa_i$. 
Writing, $\eta_{LL} = \kappa_1$, $\eta_{RL} = \kappa_3 / 2$, $\eta_{RR} = \kappa_5$, and $\kappa_2 = \kappa_4 = \kappa_6 = 0$, we shall
consider the models in Table~\ref{tab:models}.

\begin{table}[htp]
\caption{Models to be considered in this analysis}
\label{tab:models}
\medskip
 \begin{tabular}{l|lll}
 \hline
 Model	&	$\eta_{LL}\quad$	&  $\eta_{RL}\quad$	& $\eta_{RR}\quad$ \\ \hline \hline
 LL		& 	$\pm 1$	 	& 0			& 0 \\
 RR		& 	0	 		& 0			& $\pm 1$ \\
 VV		& 	$\pm 1$	 	& $\pm 1$		& $\pm 1$ \\
 AA		& 	$\pm 1$	 	& $\mp 1$		& $\pm 1$ \\
 V-A		& 	0	 		& $\pm 1$			& 0 \
\end{tabular}
\end{table}
At next-to-leading order,  the inclusive jet $p_\textrm{T}$ cross section per bin can be written as~\cite{bib:CIJET},
\begin{align}
	\sigma 	&= \sigma_\textrm{QCD} \nonumber\\
			&+ \lambda \sum_{i=1}^6 \kappa_i (b_i + a_i g + a_i f) \nonumber\\
			&+ \lambda^2 \sum_{i=1}^6 \kappa_i^2 (b_{ii} + a_{ii} g + a_{ii} f) \nonumber\\
			&+ \lambda^2 \sum_{i=1,3,5} \kappa_i \kappa_{i+1} (b_{ii+1} + a_{ii+1} g + a_{ii+1} f) \nonumber\\
			&+ \lambda^2 \sum_{i=1,2,5,6} \kappa_i \kappa_{4} (b_{i4} + a_{i4} g + a_{i4} f),		
	 \label{eq:sigma}
\end{align}
where $\sigma_\textrm{QCD}$ is the QCD cross section, $r = \ln(\Lambda/\mu_0)$ is split into two parts denoted by $g = - \ln(\mu_0 \sqrt{k})$ and
$f(\lambda) = \ln(\sqrt{k/\lambda})$, $\mu_0$ is an arbitrary bin-dependent reference scale (provided by {\tt CIJET}) and $k$ is an arbitrary bin-independent scale that we are free to choose. At leading order (LO), the $a$ terms vanish.
We shall use {\tt CIJET} to calculate the coefficients $b$ and $a$ for a given
PDF set and member. Since the cross section is linear in the coefficients $b$, $a \times g$, and $a$, we can smear these coefficients independently of the model parameters $\lambda$, $\kappa_1,\cdots,\kappa_6$ (see Sec.~\ref{sec:smearing}). This will allow us to
construct a family of  models in which the 7 parameters appear explicitly.

These models are defined using parton-level jets, whereas CMS corrects observed jets so that,
on average, their energy scale matches that of jets at the hadron (i.e, particle) level. Therefore, in principle, the models must be corrected
to the hadron level in order to compare data with theory. This non-perturbative (NP) correction,
which currently cannot be calculated from first principles, is applied as a multiplicative    
factor to the parton-level inclusive jet differential cross section. The NP correction 
\begin{align}
	C_\textrm{NP}(\pT) & = A + B / p_\textrm{T}^n, \nonumber\\
		 A &= 1.003, \nonumber\\
		 B &= 77.374, \nonumber\\
		 n &= 1.385,
\end{align}
has been derived by the CMS Inclusive Jet measurement group by
comparing parton-level jet predictions with predictions at the hadron level, computed
using  
{\tt PYTHIA}.  In addition to the non-perturbative corrections, we also include 
the far more important electroweak corrections provided by the Inclusive Jet measurement
group. These corrections rise to about 12\% at jet $p_T \sim$2TeV.

\section{Observations}
The high-$p_\textrm{T}$ end of the observed inclusive jet spectrum
(CMS PAS SMP-12-012, CMS Analysis Note AN2012\_223\_V16, 2013 and its
updates) with the full 8TeV integrated luminosity of 19.71fb$^{-1}$
and the Winter 2014, V8, jet energy corrections (JEC) is shown in 
Table~\ref{tab:yield}.

\input{data}


\section{Jet Response}
\label{sec:smearing}

As alluded to above, we shall not unfold the spectrum. Instead, we shall convolve the predicted spectra using
the jet response function 
\begin{equation}
R(\pTprime| z) = \textrm{Gaussian}(\pTprime, z,  \sigma_z),
\label{eq:R}
\end{equation}
where $\pTprime$ is the observed jet transverse momentum using the \emph{nominal}
jet energy scale, 
$z$ is the true jet transverse momentum and $\sigma_z$, the jet  transverse momentum resolution, is  
 given by 
\begin{equation}
	\sigma_z = z C_{Data} \sqrt{\frac{N^2}{z^2} + \frac{S^2}{z} + C^2}, 
	\label{eq:JER}
\end{equation}
where $C_{Data} = 1.052$, $N = 5.7936$ GeV, $S = 0.984$ GeV$^{1/2}$, and
$C = 0.029$. 
These constants
(derived from simulated jets) are for the rapidity bin $|y| < 0.5$.  As is clear from Table~6 in
AN2012\_223\_V16, 2013, the jet  resolution depends slightly on rapidity.
%\footnote{There is a typo in Table~6; the numbers for $N$ and $C$ are switched.}.

Equation~(\ref{eq:R}) is the appropriate response function for the nominal jet energy scale. 
In order to account for uncertainty in the JES and JER, we introduce two zero mean, unit
variance, Gaussian variates  $x$ and $y$, respectively.  
We account for the 10\% uncertainty in the JER, $\sigma_z$, by scaling it by the
factor $Y = 1 + y\sigma_y$. Likewise, we account for the \pT-dependent uncertainty
in the JES, which varies from 1\% at $\pTprime = 500$ GeV to 1.5\% at $\pTprime = 2,500$ GeV
according to the results encoded in the uncertainty source file {\tt Winter14\_V5\_DATA\_UncertaintySources\_AK7PDF.txt}\footnote{We need to migrate to V8.}, by scaling the nominal jet \pT, \pTprime, by the 
 factor $X = 1 + x \sigma_\pTprime$.  With these modifications, the jet response function
 becomes
\begin{equation}
R(\pT| x, y, z) = \delta(\pT - X\pTprime) \textrm{Gaussian}(\pTprime, z,  Y\sigma_z). 
\label{eq:Response}
\end{equation}
The details of the JES uncertainty calculations may be found at

{\tt https://twiki.cern.ch/twiki/bin/viewauth/CMS/JECUncertaintySources}\\
 
The bin-by-bin quantities $\sigma_\textrm{QCD}$, $b$, $a \times g$, and $a$ (58 per bin) will
be converted to differential quantities by dividing them by the $p_\textrm{T}$ bin width and the rapidity bin width $|y| = 0.5$.  The differential quantities will
then be convolved with the jet energy response function, Eq.~(\ref{eq:Response}), for 
a random pair of values $(x, y)$. This will yield smeared differential distributions,
 \begin{equation}
	f_\textrm{obs}(\pT | x,  y) =   \int_0^\infty dz \,  R(\pT | x, y, z) \, f(z),
	\label{eq:cj}
\end{equation}
for each of the 58  coefficients  in Eq.~(\ref{eq:sigma}),
where $f(z)$ represents a smooth interpolation of a differential distribution. The ensemble
of functions $\{ f_\textrm{obs}(\pT | x,  y) \}$ will encode the uncertainty in both the JES and the JER.

As noted, the cross section per bin, Eq.~(\ref{eq:sigma}), is a linear combination of 58 terms. Therefore, for every PDF set and member within that set we must 
compute 58 smeared differential distributions, each of which must then be integrated over
 20 jet \pT\ bins.  The spectra from the three PDF sets will be pooled into a single ensemble of
smeared spectra that will represent the distribution of spectra that incorporates smearing due to
PDF uncertainties as well as jet smearing.  In order to separate out the effect of jet smearing from
that due to PDF uncertainties, we shall repeat the procedure using the nominal PDFs, that is, the PDFs denoted by PDF member zero in each PDF set. 
%
\section{Roadmap}
In order to permit integration over the PDF parameters, we shall 
generate a sample of PDF set members and place them in the standard LHAPDF area for PDF sets ({\tt share/lhapdf/PDFsets}). Note: samples already exists for NNPDF. 
	
We shall make use of two programs from the {\tt CIJET} package, {\tt dijets4ci\_mul} and {\tt ciconv}. The
first program must be run in a directory at the same level as the directory {\tt fastCI}, which 
contains the sub-directory {\tt fgrid}. Here is a suggested directory structure:
\begin{verbatim}
work/
	data/			run dijets4ci_mul here
				output from dijets4ci_mul (*_fitresults.dat files)	
	fastCI/fgrid/	output from dijets4ci_mul (*_app1.sum files)
			
	fastCI/		run ciconv here
		  CT10/	output from ciconv
		  	   000/
			   :
			   099/
		  MSTW/
		  	   000/
			   :
			   099/
		  NNPDF/
		  	   000/
			   :
			   099/	
\end{verbatim}
 The calculation of cross sections using {\tt CIJET} is
a two-step process. First, for each true jet $p_\textrm{T}$ bin, 
one calculates a set of coefficients using the program {\tt dijets4ci\_mul}:
\begin{verbatim}
	dijets4ci_mul <tagname>
	Example:
	dijets4ci_mul ci0507 
\end{verbatim}
The results of this step, which can take several hours, is a file {\tt ci0507\_fitresults.dat} in the 
{\tt data} directory, which are  independent of the PDF set, renormalization and factorization scales
($\mu_r$ and $\mu_f$, 
respectively), and of the model parameters $\Lambda$, $\kappa_1,\cdots,\kappa_6$. Therefore, they need be calculated only once per center-of-mass energy and $p_\textrm{T}$ and rapidity bin. 

Next, the program {\tt ciconv}, which is run from the {\tt fastCI} directory, is used to compute coefficients for a specific PDF set/member and 9
different pairs of scale choices. Given these coefficients, one can calculate the cross
section for any values of the model parameters for the 9 different pairs of scale choices:
\begin{verbatim}
	ciconv <PDF-set-name>  <PDF-number >  <fgrid-filename>   <output-filename>
	Example:
	ciconv CT10nlo 0 ci0507_app1.sum CT10/000/ci0507.txt
\end{verbatim}
In the example, we are calculating coefficients for member 0 of CT10nlo using the 
coefficients stored in the {\tt grid} file {\tt fastCI/fgrid/ci0507\_app1.sum}, with the results written
to file {\tt CT1/000/ci0507.txt}. This is to be repeated for each of the PDF members.


\subsection{Likelihood}

We shall follow the 7 TeV analysis and use the  multinomial likelihood function
\begin{equation}
p(D|\lambda, \kappa_1,\cdots,\kappa_6, \omega) = \binom{N}{N_1,\cdots,N_K} \prod_{i=1}^K \left(\frac{\sigma_i}{\sigma}\right)^{N_i},
\label{eq:like}
\end{equation}
where $\sigma \equiv \sum_{i=1}^K \sigma_i$,
$N \equiv \sum_{i=1}^K N_i$ is the total observed count, $N_i$ the count in jet $p_\text{T}$ bin $i$, and $\omega$ denotes the nuisance
parameters,  $\sigma_\text{QCD}, b, a\times g$, and $a$, over which the likelihood will be marginalized.

\bigskip
\bigskip
\bigskip

The calculations of the QCD cross sections are done in the {\tt fastNLO} directory.

\appendix

\section{Installation}
This work requires the packages, {\tt LHAPDF-6.1.5}, {\tt fastnlo\_toolkit-2.3.1pre-1871}, and
{\tt CIJET-1.1}. 
The {\tt fastNLO} program is used to calculate the inclusive jet 
$p_\textrm{T}$ spectrum, while {\tt CIJET} is used to calculate, at next-to-leading order (NLO) accuracy,  the spectra arising from contact
interactions (CI).
The {\tt LHAPDF} package must be compiled first because it is needed by {\tt fastNLO}
and {\tt CIJET}. 

First create a work area, say {\tt CMSCI}, and copy the directories {\tt CI}, {\tt corrections}, {\tt data},
{\tt fastNLO}, and {fastCI} to {\tt CMSCI} from the {\tt Dropbox/CMSCI}. 

\newpage
Below we assume that you are using a {\tt bash} shell.

\begin{enumerate}
\item
{\bf Create} an area into which the codes will be installed:
	\begin{verbatim}
	cd
	mkdir external
	cd external
	mkdir bin
	mkdir lib
	mkdir include
	mkdir downloads
	mkdir -p share/LHAPDF/PDFsets
	\end{verbatim}
In your {\tt .bash\_profile} file, which is a hidden file in your home directory, you should update the environment variables {\tt PATH} and {\tt LD\_LIBRARY\_PATH} 
as follows. (I'm assuming that you have already setup {\tt Root}.)
	\begin{verbatim}
	export PATH=\$HOME/external/bin:$PATH
	export LD_LIBRARY_PATH=\$HOME/external/bin:\$LD_LIBRARY_PATH
	\end{verbatim}

Next time you create a terminal window these variables will be updated. For now, just do
	\begin{verbatim}
	cd 
	source .bash_profile
	\end{verbatim}

Below we assume that the files to be unpacked are in the directory 

{\tt \$HOME/external/downloads}

\item {\bf Install} {\tt LHAPDF-6.1.5}.  {\tt NB:} There is a bug in this version of {\tt LHAPDF}. It will be
fixed in version 6.1.6. In the meantime, to fix
the bug do the following. Wherever
you see the {\tt abs} function in the source code {\tt LHAPDF-6.1.5/src/PDFSet.cc}, simply 
delete the word {\tt abs} and nothing else and save the corrected source code. 
	\begin{verbatim}
	cd 
	cd external
	tar zxvf downloads/LHAPDF-6.1.5.tar.gz
	cd LHAPDF-6.1.5
	./configure --prefix=$HOME/external
	make
	make install
	\end{verbatim}
Next, we need to compile the programs in the directory {\tt LHADPDF-6.1.5/examples}, in 
particular, {\tt hessian2replicas}, which we shall use to generate a random sample of 500 PDF sets
from {\tt CT10nlo}, and later from the {\tt MSTW} PDF set.
	\begin{verbatim}
	cd 
	cd external/LHAPDF-6.1.5/examples
	make
	
	cd 
	cd external/bin
	ln -s ../LHAPDF-6.1.5/examples/hessian2replicas .
	\end{verbatim}
In the last step, we created a link in {\bf bin} to {\tt hessian2replicas}.
\smallskip

In order to use {\tt LHAPDF}, it is necessary to set the environment  variable
{\tt LHAPDF\_DATA\_PATH} to the directory that contains the PDF sets; for example:
\begin{verbatim}
	export LHAPDF_DATA_PATH=$HOME/external/share/LHAPDF/PDFsets
\end{verbatim}
Also remember to execute the {\tt setup.sh} script, which should be copied from the {\tt Dropbox} and placed in your working directory, assumed to be called {\tt CMSCI}. 
	\begin{verbatim}
	cd CMSCI
	source setup.sh
	\end{verbatim}
\item {\bf Install} {\tt fastnlo\_toolkit-2.3.1pre-1871}. 

	\textcolor{red}{{\bf Important}}: {After you have unpacked this file, you should replace the program
	{\tt fnlo-tk-cppread.cc} in the {\tt fastnlo\_toolkit-2.3.1pre-1871/src} with the version that
	is in {\tt Dropbox/CMSCI/downloads}. The version in the {\tt Dropbox} contains a version
	with an extra argument for the PDF member. Also check that you have the most recent version of {\tt runfastNLO.py} (15-Apr-2015) in your {\tt fastNLO/work} area.}
	\begin{verbatim}
	cd
	cd external 
	tar zxvf downloads/fastnlo_toolkit-2.3.1pre-1871.tar.gz
	cd fastnlo_toolkit-2.3.1pre-1871
	./configure --prefix=$HOME/external 
	            --enable-pyext
	make
	make install
	\end{verbatim}

\item {\bf Unpack} and compile {\tt CIJET-1.1}.
	\begin{verbatim}
	cd
	cd external
	tar zxvf downloads/CIJET-1.1.tar.gz
	cd CIJET1.1
	make
	\end{verbatim}
	We shall be using the {\tt CIJET} program {\tt ciconv}. In order to make it globally accessible 
	create a link to it as follows
	\begin{verbatim}
	cd
	cd external/bin
	ln -s ../CIJET1.1/fastCI/ciconv .
	\end{verbatim}
	
\item {\bf Download} PDF Sets

For  now, just download the PDF set {\tt CT10nlo}. Later, download the other two sets.
	\begin{verbatim}
	cd
	cd external/share/LHAPDF/PDFsets
	wget https://www.hepforge.org/archive/lhapdf/pdfsets/6.1/CT10nlo.tar.gz	
	tar zxvf CT10nlo.tar.gz
	\end{verbatim}

\item {\tt Generate} randomly sampled PDF sets
	\begin{verbatim}
	cd
	cd external/share/LHAPDF/PDFsets
	hessian2replicas CT10nlo 5678 500
	\end{verbatim}
This will create and new directory in {\tt external/share/LHAPDF/PDFsets} called 
{\tt CT10nlo\_rand5678} with 500 randomly sampled PDFs from {CT10nlo}. The random
number seed is 5678. Each of these 500 PDF members (plus the the nominal PDF member identified as member zero) will be used to calculate a QCD and CI
spectrum. The spread in these spectra will reflect the uncertainty in the PDFs.

\end{enumerate}

We are now, finally, able to start our calculations! 

\section{Computing Spectra}

\subsection{Computing unsmeared QCD spectra}

\textcolor{black}{
We compute QCD spectra for each of the 501 PDF members of CT10nlo using 7 different 
combinations of renormalization and factorization scales. These calculations are performed with
the program {\tt fnlo-tk-cppread}, which is called by the script {\tt fastNLO/work/runfastNLO.py},
which in turn is called as follows}
	\begin{verbatim}
	cd CMSCI/fastNLO/work
	python runfastNLO.py CT10nlo_rand5678
	\end{verbatim}

This will create 7 text files per PDF member in the directories labeled
	\begin{verbatim}
	../CT10/NNN
	\end{verbatim}
where {\tt NNN} is the PDF member number (000 to 500).  Once this is done, we convert the text files into {\tt Root} histograms stored in files as follows:
	\begin{verbatim}
	python createQCDhists.py CT10
	\end{verbatim}

\subsection{Computing smeared QCD spectra}

\textcolor{black}{
The next step is to smear the theoretical spectra with the CMS jet response function. 
First, compile the relevant code in the {\tt CI} directory under {\tt CMSCI}:}
	\begin{verbatim}
	cd CI
	make
	\end{verbatim}
{\bf Note}: If your {\tt makefile} fails, you can try deleting the code fragments 
	\begin{verbatim}
	>& $*.FAILED
	\end{verbatim}
which appear in two places in the {\tt Makefile}, then try {\tt make} again.

Now, go to the main {\tt work} directory under {\tt CMSCI}, where you will find the script
{\tt smearSpectra.py}. In order to smear the spectra in the directory {\tt ../fastNLO/CT10} do
	\begin{verbatim}
	python smearSpectra.py ../fastNLO/CT10
	\end{verbatim}

This will create new {\tt qcd.root} files, containing the smeared QCD spectra, in the directories
{\tt fastNLO/CT10/NNN/JESJERPDF}, where {\tt NNN} is the PDF member number.

\subsection{Computing unsmeared CI coefficient spectra}

As noted above, the calculation of the CI spectra proceeds in two steps. The first step takes several
hours. However, it need be done only once for a given center-of-mass energy and  $p_\text{T}$
and $\eta$ bin. Thereafter, the calculations are fast.

For each of the three PDF sets we proceed
as follows: 
	\begin{enumerate}
	\item Move to the {\tt data} directory and run the {\tt CIJET} program {\tt dijets4ci\_mul <tag-name>}. This will  create output in {\tt data} as well as 
	in {\tt fastCI/fgrid}. Remember, the directories {\tt data} and {\tt fastCI} must be at the same level. In practice, this calculation should be run as a batch job, one for each \emph{true}  jet $p_\text{T}$ bin. However, since the bins of true jet $p_\textrm{T}$ are specified in 
	{\tt bininput.card},
	the kinematics in {\tt kininput.card}, and the process (inclusive jets) in {\tt proinput.card} (see Appendix), it will be necessary for each batch job to use a different ``{\tt data}" 
	directory for each jet $p_\text{T}$
	bin. I suggest, therefore, that you create a different {\tt data} directory for each jet $p_\text{T}$, labeling the {\tt data} directories {\tt dataNNNN} where {\tt NNNN} is the  $p_\text{T}$
	bin (see the example {\tt data0049} in the dropbox). Remember to set the bin in
	{\tt bininput.card}.  In the example in the dropbox, the $p_\text{T}$ bin is 49 to 56. To compute 
	the coefficients needed for subsequent  calculations of the CI cross sections, one does:
	\begin{verbatim}
	cd data0049
	dijets4cl_mul ci0049
	\end{verbatim}
	
	\item 
	Once the above time-consuming calculations have been performed for every {\tt true} jet $p_\text{T}$ bin, we can use
	their results to compute for every PDF set and member  the coefficients needed to
	calculate cross sections. Then, these coefficient spectra can be used to calculate differential
	cross sections for any values of the parameters, $\Lambda$ and $\kappa_1,\cdots,\kappa_6$.
	
	We compute CI coefficient spectra for each PDF member of CT10nlo using 9 different 
combinations of renormalization and factorization scales. The calculations are performed with
the program {\tt ciconv}, which is called by the Python program {\tt fastCI/runciconv.py} and which must be run
in the {\tt fastCI} directory. The program {\tt runciconv.py} loops over all PDF members and, for each,
over all the grid files {\tt fgrid/ci*.sum} and calls {ciconv} for a given PDF set, PDF member, and grid file.
The results are written to the directory {\tt CT10}, which contains a subdirectory for each PDF member.
	\end{enumerate}

\subsection{Computing smeared CI coefficient spectra}
Just as for QCD, go to the main {\tt work} directory under {\tt CMSCI} and run the script
{\tt smearSpectra.py},
	\begin{verbatim}
	python smearSpectra.py ../fastCI/CT10
	\end{verbatim}
\noindent
to create new {\tt *.root} files, one for each CI coefficient, containing the smeared coefficient spectra, in the directories
{\tt fastCI/CT10/NNN/JESJERPDF}, where {\tt NNN} is the PDF member number.  This calculation  takes
about two hours on a laptop.

\subsection{Creating a {\tt RooFit} workspace}
The last step before limit calculations is to create a workspace in which the results of the previous
calculations are stored. This is done as follows:
\begin{verbatim}
python createWorkspace.py CT10
\end{verbatim}
This creates a {\tt RooWorkspac} called {\tt CT10\_JESJERPDF\_workspace.root}, which is the basis
for the limit calculations.


You can check the integrity of the workspace by running the workspace plotting program:
\begin{verbatim}
python plotWorkspace.py -m LL CT10_JESJERPDF_workspace.root
\end{verbatim}
This will display the spectra for a Left-Left (LL) CI model with $\Lambda = 20\,\text{TeV}$ for both
constructive and destructive interference. If this is successful, we shall be ready to calculate limits.

\section{Input Files for {\tt CIJET}}

{\bf Example {\tt bininput.card}}
	\begin{verbatim}
massbin         	[inv. mass or pt bin list in GeV]
507 548
rapbin          	[angular bin list, chi, ystar, ymax or |y|]
0.0 0.5
	\end{verbatim} 
	
{\bf Example {\tt kininput.card}}

	\begin{verbatim}
jetscheme	1		[jet algorithm, 1 for kT (=CA, anti-kT), 2 for others]
recscheme	2		[jet recombination scheme, 1 for "ET", 2 for "4-vector(LHC)"]
Rcone		0.7		[cone size or distance parameter]
Rsep		1.3		[Rsep used in midpoint algorithm]
ptcut		20		[acceptance condition for jet pT/GeV]
ycut		2.5		[acceptance condition for jet rapidity]
ptadd1		20		[additional cut on pT/GeV of jet1 (harder one) in dijet]
ptadd2		20		[additional cut on pT/GeV of jet2 (softer one) in dijet]
yb		1.11		[upper cut on yboost of dijet system]
ys		1.39		[upper cut on ystar of dijet system]
	\end{verbatim} 
	
{\bf Example {\tt proinput.card}}

\begin{verbatim}
pdf  CT10nlo.LHgrid    [pdf name as in LHAPDF]
pdfmember  0   [pdf member as in LHAPDF]
mode		fital                   [fit "fitcs/ll/al"]
pseed		2444                [seed of the random number generator]
many		1000000  [typical number of MC points, rec. 3M or adjust accordingly]
ang			4      [1 for chi, 2 for ystar, 3 for ymax, 4 for inclusive jet]
ppcollider		1       [1 for LHC, 0 for Tevatron]
sqrtS			8000    [c.m.s. energy of collider, in GeV]
scalescheme	0                       [choice of dynamic scales]
\end{verbatim}


\begin{thebibliography}{99}
\bibitem{bib:7TeVCI}
CMS Collaboration, Chatrchyan, S. et al., 
``Search for contact interactions using the inclusive jet pT spectrum in pp collisions at $\sqrt{s}$=7 TeV",  Phys.Rev. {\bf D87} (2013) 5, 052017;
CMS-EXO-11-010, CERN-PH-EP-2013-002,
DOI: 10.1103/PhysRevD.87.052017,
e-Print: arXiv:1301.5023.

\bibitem{bib:MSTW}
JHEP 08 (2012) 052 [arXiv:1205.4024 [hep-ph]] and {\tt http://mstwpdf.hepforge.org/random}.
 
\bibitem{bib:Gao}
J. Gao et al., ``Next-to-leading QCD effect to the quark compositeness search at the LHC",
Phys. Rev. Lett. {\bf 106} (2011) 142001, {\tt arXiv:1101.4611}.
\bibitem{bib:CIJET}
J. Gao, CIJET, {\tt arXiv:1301.7263}.
\bibitem{bib:CMSCI12}
CMS Collaboration, Chatrchyan, S. et al., ``Search for quark compositeness in dijet angular distributions from pp collisions at $\sqrt{s}=7$ TeV",  JHEP {\bf 1205} (2012) 055 (CMS-EXO-11-017). 
\end{thebibliography}
\end{document}